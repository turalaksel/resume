%!TeX program = pdflatex
% Tural Aksel's Curriculum Vitae
% Email: turala@nautilus.bio
% Web: https://turalaksel.com/
% Repo: https://github.com/turalaksel/cv

\documentclass[12pt,letterpaper]{report}

\usepackage[T1]{fontenc} % output T1 font encoding (8-bit) for accented characters as single glyph
\usepackage[strict,autostyle]{csquotes} % smart and nestable quote marks
\usepackage[USenglish]{babel} % regionalize hyphens, quote marks, etc automatically
\usepackage{microtype}% improve text appearance with kerning, etc
\usepackage{datetime} % enable formatting of date output
\usepackage{tabto}    % make nice tabbing
\usepackage{hyperref} % enable hyperlinks and pdf metadata
\usepackage{geometry} % manually set page margins
\usepackage{enumitem} % enumerate with [resume] option
\usepackage{titlesec} % allow custom section fonts
\usepackage{setspace} % custom line spacing
\usepackage{makecell} % for line breaks in table format

% what is your name?
\newcommand{\myname}{Tural Aksel}

% select default typefaces
\usepackage{ebgaramond} % document's serif typeface
\usepackage{helvet}     % document's sans serif typeface

% how far to tab for list items with left-aligned date: different fonts need different widths
\newcommand{\listtabwidth}{1.7cm}

% define font to use as document's title
\newcommand{\namefont}[1]{{\normalfont\bfseries\Huge{#1}}}

% set section heading fonts and before/after spacing
\SetTracking{encoding=*, family=\sfdefault}{30} % increase sans serif headings tracking
\titleformat{\section}{\lsstyle\sffamily\small\bfseries\uppercase}{}{}{}{}
\titlespacing{\section}{0pt}{30pt plus 4pt minus 4pt}{8pt plus 2pt minus 2pt}

% set subsection heading fonts and before/after spacing
\titleformat{\subsection}{\lsstyle\sffamily\footnotesize\bfseries}{}{}{}{}
\titlespacing{\subsection}{0pt}{16pt plus 4pt minus 4pt}{4pt plus 2pt minus 2pt}

% set page margins (assumes letter paper)
\geometry{body={6.5in, 9.0in},
    left=1.0in,
    top=1.0in}

% prevent paragraph indentation
\setlength\parindent{0em}

% set line spacing
\setstretch{0.9}

% define space between list items
\newcommand{\listitemspace}{0.25em}

% make unordered lists without bullets and use compact spacing
\renewenvironment{itemize}
{\begin{list}{}{\setlength{\leftmargin}{0em}
                \setlength{\parskip}{0em}
                \setlength{\itemsep}{\listitemspace}
                \setlength{\parsep}{\listitemspace}}}
{\end{list}}

% Update cell alignment
\renewcommand{\cellalign}{cl}

% make tabbed lists so content is left-aligned next to years
\TabPositions{\listtabwidth}
\newlist{tablist}{description}{3}
\setlist[tablist]{leftmargin=\listtabwidth,
    labelindent=0em,
    topsep=0em,
    partopsep=0em,
    itemsep=\listitemspace,
    parsep=\listitemspace,
    font=\normalfont}

% print only the month and year when using \today
\newdateformat{monthyeardate}{\monthname[\THEMONTH] \THEYEAR}

% define hyperlink appearance and metadata for pdf properties
\hypersetup{
    colorlinks  = true,
    urlcolor    = black,
    pdfauthor   = {\myname},
    pdfkeywords = {DNA nanotechnology, Cryo-EM, protein design and engineering, molecular motors, single molecule force spectroscopy, protein folding},
    pdftitle    = {\myname: Curriculum Vitae},
    pdfsubject  = {Curriculum Vitae},
    pdfpagemode = UseNone
}

\begin{document}
    \raggedright{}

    % display your name as the document title
    \namefont{\myname}

    % affiliation and contact info blocks
    \vspace{1em}
    \begin{minipage}[t]{0.700\textwidth}
        % current primary affiliation, left-aligned
        Nautilus Biotechnology \\
        201 Industrial Rd \#310 \\
        San Carlos, CA \\
    \end{minipage}
    \begin{minipage}[t]{0.295\textwidth}
        % contact info details, right-aligned
        \flushright{}
        \href{mailto:turalaksel@ngmail.com}{turalaksel@gmail.com} \\
        +1 410 369 6615 \\
        \href{https://www.linkedin.com/in/turalaksel}{turalaksel.com}
        \href{https://github.com/turalaksel}{github.com/turalaksel}
    \end{minipage}

    \section*{Research Interests}

    \begin{itemize}

        \item \textbullet \hspace{0.2cm} Development of DNA nanotechnology tools for structural biology, proteomics, immunotherapy and bioenergy production. 
        \item \textbullet \hspace{0.2cm} Scientific software development for biomolecular design, image processing and data analysis.
        \item \textbullet \hspace{0.2cm} Protein engineering and design for hybrid DNA Origami-protein complexes.

    \end{itemize}

    \section*{Education}

    \begin{tablist}

        \item[Ph.D.] \tab{}Biophysics \hfill 2012\\
        \textbf{Johns Hopkins University}, Baltimore, MD\\
        \textit{Thesis Advisor}: Doug Barrick 
        
        \item[B.S.]  \tab{}Biological Sciences and Bioengineering \hfill 2006\\
        \textbf{Sabanci University}, Istanbul, Turkey\\
        \textit{Thesis Advisor}: Ugur Sezerman

    \end{tablist}

    \section*{Professional \& Academic experience}

    \begin{tablist}

        \item[2019--]   \tab{}Nautilus Biotechnology, San Carlos \\
                             \textbf{Senior Scientist}, DNA Nanotechnology\\
                             \begin{itemize}
                                \item \textbullet \hspace{0.2cm} I lead a team to develop DNA Origami devices for proteomics research. I direct day-to-day and long term research activities of my team members.
                                \item \textbullet \hspace{0.2cm} I have developed the key DNA Origami technologies for Nautilus platform.
                                \item \textbullet \hspace{0.2cm} My research achievements have led to three patent applications as the lead inventor. 
                             \end{itemize}

        \item[2018--20] \tab{}University of California, San Francisco \\
                             \textbf{Applications Programmer III} \\
                             \textit{PI}: Shawn Douglas \\
                             \begin{itemize}
                                \item \textbullet \hspace{0.2cm} I developed a DNA Origami platform and image processing pipeline on AWS cloud for high-resolution cryo-EM studies of small proteins. The technology enables structural studies of small DNA binding proteins that wouldn't be otherwise studied using conventional cryo-EM. The method is published in Nature Biotechnology. 
                                \begin{itemize}
                                    \item \hspace{1cm} \textbf{Publication:} \href{https://doi.org/10.1038/s41587-020-0716-8}{Aksel T et al.(2021) \textit{Nature Biotechnology}}. 
                                    \item \hspace{1cm} \textbf{Cryoorigami software package:} \href{https://github.com/douglaslab/cryoorigami} {github.com/douglaslab/cryoorigami}. 
                                \end{itemize}
                                \item \textbullet \hspace{0.2cm} I developed new methods and software for 1) Thermodynamically optimized DNA Origami designs, and 2) DNA Origami structure prediction. The tools will be made publicly available in a webserver (in progress).
                             \end{itemize} 

        \item[2015--18]\tab{}University of California, San Francisco \\
                             \textbf{Postdoctoral Fellow}, Department of Cellular and Molecular Pharmacology \\
                             \textit{PI}: Shawn Douglas
                             \begin{itemize}
                                \item \textbullet \hspace{0.2cm} I worked on the development of a DNA nanotechology platform for high-resolution cryo-EM studies of small proteins.
                                \item \textbullet \hspace{0.2cm} I developed a scalable technology for the production of custom DNA Origami scaffolds.
                                \item \textbullet \hspace{0.2cm} I designed a DNA Origami structure for tunable activation of Car-T cells. The DNA Origami design and the results for the publication are published in PNAS.
                                \begin{itemize}
                                    \item \hspace{1cm} \textbf{Publication:} \href{https://doi.org/10.1073/pnas.2109057118}{Dong R, Aksel T et al.(2021) \textit{PNAS}}
                                \end{itemize}
                                \item \textbullet \hspace{0.2cm} I designed a chimeric adapter protein for the display of non-DNA binding proteins on our DNA Origami platform (in progress).
                            \end{itemize}

        \item[2013--15]\tab{}Stanford University \\
                             \textbf{Postdoctoral Fellow}, Biochemistry Department \\
                             \textit{PI}: James Spudich \\
                             \begin{itemize}
                                \item \textbullet \hspace{0.2cm} I developed a loaded actin gliding assay to quantify the power output generated by cardiac myosins.
                                \item \textbullet \hspace{0.2cm} I developed an image processing software for automated filament tracking. The assay and the filament tracking software helped us quantify the power output generated by cardiac myosin mutants.   
                                \begin{itemize}
                                    \item \hspace{1cm} \textbf{Publication:} \href{https://doi.org/10.1016/j.celrep.2015.04.006}{Aksel T et al.(2015) \textit{Cell Reports}}. 
                                    \item \hspace{1cm} \textbf{FASTrack filament tracking software:} \href{https://github.com/turalaksel/FASTrack} {github.com/turalaksel/FASTrack}. 
                                \end{itemize}
                             \end{itemize}  

        \item[2006--12]\tab{}Johns Hopkins University \\
                             \textbf{Ph.D. student}, Department of Biophysics\\
                             \textit{PI}: Doug Barrick \\
                             \begin{itemize}
                                \item \textbullet \hspace{0.2cm} I studied the origins of cooperativity and pathway diversity in protein folding using consensus Ankyrin repeat proteins (CARPs). I generated CARPs from identical consensus Ankyrin repeat units by a modular cloning method. 
                                \begin{itemize}
                                    \item \hspace{1cm} \textbf{Publication:} \href{https://doi.org/10.1016/j.str.2010.12.018}{Aksel T et al.(2011) \textit{Structure}}
                                \end{itemize} 
                                \item \textbullet \hspace{0.2cm} I developed a nearest-neighbor statistical physical model called Ising model to dissect folding energetics into individual repeat stability and repeat-repeat interface terms for repeat proteins from experimental data. I developed a python package to fit the Ising model to a series of equilibrium and kinetic folding data to determine the folding energy for single repeat folding and repeat-repeat interface formation.
                                \begin{itemize}
                                    \item \hspace{1cm} \textbf{Publication:} \href{https://doi.org/10.1016/S0076-6879(08)04204-3}{Aksel T et al.(2009) \textit{Methods in Enzymology}}
                                    \item \hspace{1cm} \textbf{Isingbul data fitting software:} \href{https://github.com/turalaksel/IsingBul1.0} {github.com/turalaksel/IsingBul1.0}.
                                \end{itemize} 

                                \item \textbullet \hspace{0.2cm} I developed an efficient software, written in C++, to calculate the 3D Ising Model partition function for biological systems. I used this tool to predict the pKa values of titratable residues from protein structure.
                            \end{itemize} 
        
        \item[2006]\tab{}Sabanci University, Istanbul, Turkey \\
                             \textbf{Instructor} Computer Science Department\\
                             \textit{Course}: Data Structures
                             \begin{itemize}
                                \item \textbullet \hspace{0.2cm} I taught the summer school Data structures course in computer science department.
                                \item \textbullet \hspace{0.2cm} I developed an homology model algorithm for structure prediction of protein sequences. The algorithm recursively finds the best matching patterns between two protein sequences using dynamic algorithm.  
                            \end{itemize}

    \end{tablist}



    \section*{Programming Skills}
    \begin{itemize}
        \item \textbullet \hspace{0.2cm} Computing Environments: Matlab, IPython, Scilab, R, AWS clound computing.
        \item \textbullet \hspace{0.2cm} Languages: Python, C, C++, Perl, Shell scripting.
        \item \textbullet \hspace{0.2cm} Operating Systems: Unix/Linux, Windows, Mac OS.
        \item \textbullet \hspace{0.2cm} Biomolecular Modeling: Pymol, PyRosetta, Cadnano.
    \end{itemize}

    \section*{Laboratory Skills}
    \begin{itemize}
        \item \textbullet \hspace{0.2cm} Bioconjugation.
        \item \textbullet \hspace{0.2cm} DNA Nanotechnology, DNA Origami design, production and scale-up.
        \item \textbullet \hspace{0.2cm} Cryogenic electron microscopy (cryo-EM), negative-stain TEM.
        \item \textbullet \hspace{0.2cm} Recombinant DNA technologies, bacterial and mammalian protein expression, protein chromatography.
        \item \textbullet \hspace{0.2cm} CD and fluorescence spectroscopy, biomolecular NMR, SAXS/WAXS, analytical ultracentrifugation, stopped-flow kinetics.
        \item \textbullet \hspace{0.2cm} Single molecule force spectroscopy, fluorescence microscopy.
    \end{itemize}

    \section*{Selected Publications}

    \subsection*{Journal Articles}
    For Complete list of publications, please see \href{https://scholar.google.com/citations?user=K91MviEAAAAJ&hl=en}{Google scholar}\\

    \begin{tablist}

        \item[2021] \tab{}Dong R, \textbf{Aksel T}, Chan W, Germain RN, Vale RD, Douglas SM \enquote{DNA origami patterning of synthetic T cell receptors reveals spatial control of the sensitivity and kinetics of signal activation.} \textit{Proc. Natl. Acad. Sci. U. S. A.} 118 (40) e2109057118 \href{https://doi.org/10.1073/pnas.2109057118}{doi:10.1073/pnas.2109057118}

        \item[2021] \tab{}\textbf{Aksel T}, Yu Z, Cheng Y , Douglas SM \enquote{Molecular goniometers for single-particle cryo-EM of DNA-binding proteins.} \textit{Nature Biotechnology} 39 (3):378--386. \href{https://doi.org/10.1038/s41587-020-0716-8}{doi:10.1038/s41587-020-0716-8}

        \item[2015] \tab{}\textbf{Aksel T}, Yu EC, Sutton S, Ruppel KM, Spudich JA. \enquote{Ensemble Force Changes that Result from Human Cardiac Myosin Mutations and a Small-Molecule Effector.} \textit{Cell Reports} 11 (6):910--920. \href{https://doi.org/10.1016/j.celrep.2015.04.006}{doi:10.1016/j.celrep.2015.04.006}

        \item[2011] \tab{}\textbf{Aksel T}, Majumdar A, Barrick D. \enquote{The contribution of entropy, enthalpy, and hydrophobic desolvation to cooperativity in repeat-protein folding.} \textit{Structure} 19 (3):349--360 \href{https://doi.org/10.1016/j.str.2010.12.018}{doi:10.1016/j.str.2010.12.018}

    \end{tablist}
    
    \section*{Patents}

    \begin{tablist}

        \item[2021] \tab{}Coinventor of US Patent Application assigned to Nautilus Biotechnology, Filed 2021, Confidential.
        \item[2021] \tab{}Coinventor of US Patent Application assigned to Nautilus Biotechnology, Filed 2021, Confidential.
        \item[2020] \tab{}Coinventor of US Patent Application assigned to Nautilus Biotechnology, Filed 2020, Confidential.

    \end{tablist}

    \section*{Grants and Awards}

    \subsection*{Awards and Honors}

    \begin{tablist}

        \item[2008] \tab{}Brian Key PhD Student Travel Award.
        \item[2001--06] \tab{}High Honor Scholarship, Sabanci University. Istanbul, Turkey. Full tuition and accommodation coverage.
        \item[2001] \tab{}Ranked 62nd in Turkish university entrance exam among 1.4 million participants.
        \item[1998] \tab{}Ranked 56th in Turkish high school entrance exam among 0.5 million participants.

    \end{tablist}

    \subsection*{Grants and Fellowships}

    \begin{tablist}

        \item[2016--17] \tab{} F32 Ruth L. Kirschstein Postdoctoral Individual National Research Service Award (NIGMS:F32GM119322). 

    \end{tablist}

    \section*{Academic References}

    \begin{tabular}{lll}
        Doug Barrick&  Shawn M. Douglas & James Spudich \\
        \hline \\
        Professor and Chair of Biophysics  & Assistant Professor & Professor \\
        Johns Hopkins University  & University of California, San Francisco & Stanford University \\
        216 Jenkins Hall  & 600 16th St. & Beckman Center B400 \\
        Baltimore, MD 21218  & San Francisco, CA 94143 & Stanford, CA 94305 \\
        Phone: (410) 516-0409 & Phone: (415) 502-1947 & Phone:(650) 723-7634\\
        Email: barrick@jhu.edu  & Email: shawn.douglas@ucsf.edu & Email: jspudich@stanford.edu

    \end{tabular}

    \vspace{3em}

    \begin{tabular}{l}
        Ronald D. Vale \\
        \hline \\
        HHMI Vice President and Executive Director \\
        Janelia Research Campus \\
        19700 Helix Dr \\
        Ashburn, VA 20147 \\
        Phone: (571) 209-4000\\
        Email: valer@janelia.hhmi.org
    \end{tabular}

    \section*{Professional References}

    \begin{tabular}{lll}
        Michael Dorwart & Elvis Ikwa & Wayne Rainey\\
        \hline \\
        Director of Research &  Associate Scientist & Self employed\\
        Illumina & Nautilus Biotechnology & \\
        \makecell{\textit{Former director} \\ \textit{Nautilus Biotechnology}}&  & \makecell{\textit{Former HR Manager} \\ \textit{Nautilus Biotechnology}} \\
        Email: michaeldorwart@gmail.com & Email: elvisokiring@gmail.com & Email: wrainey929@gmail.com
    \end{tabular}



    % display today's date as Month Year after a vertical space below the end of the text
    \begin{center}
        \vfill
        Updated \monthyeardate\today
    \end{center}

\end{document}
